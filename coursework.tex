\documentclass[12pt]{article}         % the type of document and font size (default 10pt)
\usepackage[margin=1.0in]{geometry}   % sets all margins to 1in, can be changed
\usepackage{moreverb}                 % for verbatimtabinput -- LaTeX environment
\usepackage{url}                      % for \url{} command
\usepackage{amssymb}                  % for many mathematical symbols
\usepackage[pdftex]{lscape}           % for landscaped tables
\usepackage{longtable}                % for tables that break over multiple pages
\title{\Huge Title of Coursework \\[0.5in] \large CM3111: Big Data Analysis Coursework \\[2in]}  % to specify title

\author{Eilidh Southren \\[0.25in] Student ID: 1513195\\[3in]}          % to specify author(s)

\date{November 2018}
\usepackage{Sweave}
\begin{document}                      % document begins here
\Sconcordance{concordance:coursework.tex:coursework.Rnw:%
1 12 1 1 0 26 1 1 2 8 0 1 2 2 1 1 2 5 0 1 2 2 1 1 2 4 0 1 2 2 1 1 5 4 1 %
1 2 6 0 1 1 7 0 1 2 2 1 1 3 5 0 1 2 3 1 2 2 5 1 1 2 1 0 1 1 16 0 1 2 8 %
1}


% If .nw file contains graphs: To specify that EPS/PDF graph files are to be 
% saved to 'graphics' sub-folder
%     NOTE: 'graphics' sub-folder must exist prior to Sweave step
%\SweaveOpts{prefix.string=graphics/plot}

% If .nw file contains graphs: to modify (shrink/enlarge} size of graphics 
% file inserted
%         NOTE: can be specified/modified before any graph chunk
\setkeys{Gin}{width=1.0\textwidth}

\maketitle              % makes the title
\pagebreak
\tableofcontents        % inserts TOC (section, sub-section, etc numbers and titles)
%\listoftables           % inserts LOT (numbers and captions)
%\listoffigures          % inserts LOF (numbers and captions)
%                        %     NOTE: graph chunk must be wrapped with \begin{figure}, 
%                        %  \end{figure}, and \caption{}
%%%%%%%%%%%%%%%%%%%%%%%%%%%%%%%%%%%%%%%%%%%%%%%%%%%%%%%%%%%%%%%%%%%%
% Where everything else goes
\pagebreak
\section{How to typeset \textsf{R} code}

If you want to see both the input and output, do this:

\begin{Schunk}
\begin{Sinput}
> runif(10)
\end{Sinput}
\begin{Soutput}
 [1] 0.7294830 0.6171384 0.2868468 0.8641063 0.8033892 0.3448469 0.7694435
 [8] 0.9101910 0.8145743 0.9859595
\end{Soutput}
\end{Schunk}

If you want to see output, but no input, do this:

\begin{Schunk}
\begin{Soutput}
 [1] 0.25210551 0.25965330 0.26975267 0.35695254 0.08549717 0.65365451
 [7] 0.24909594 0.67897609 0.02640378 0.22510347
\end{Soutput}
\end{Schunk}

If you want to see input, but no output, do this:

\begin{Schunk}
\begin{Sinput}
> runif(13)
\end{Sinput}
\end{Schunk}

If you want to run some \textsf{R} code but hide the input/output from the reader then you can do both at the same time:


\bigskip   % leave some empty space (optional)

and you can double-check that it worked later (if you like)

\begin{Schunk}
\begin{Sinput}
> x  # use keep.source=TRUE if you want comments printed
\end{Sinput}
\begin{Soutput}
 [1]  2  3  4  5  6  7  8  9 10 11
\end{Soutput}
\begin{Sinput}
> y
\end{Sinput}
\begin{Soutput}
 [1] 0.5095516 0.6010161 0.4196946 0.5488765 0.2330574 0.4130772 0.8872260
 [8] 0.8359316 0.8522625 0.1369938
\end{Soutput}
\end{Schunk}

If you want to write some \textsf{R} code but not have it evaluated at all then do this:

\begin{Schunk}
\begin{Sinput}
> # whatever you write here must be syntactically correct R code
> runif(1000000000000000000000000)
\end{Sinput}
\end{Schunk}

If you would like to include a figure that's generated completely by \textsf{R} code, then you can do something like the following.

\begin{figure}
\includegraphics{coursework-007}
\caption{Here is the plot we made}
\end{figure}


Sometimes we would like the output to look like \LaTeX\ output instead of \textsf{R} output.  In that case, do the following.

\begin{Schunk}
\begin{Sinput}
> library(xtable)
> xtable(summary(lm(y ~ x)), caption = "Here is the table we made")
\end{Sinput}
% latex table generated in R 3.5.1 by xtable 1.8-3 package
% Fri Oct 05 22:12:01 2018
\begin{table}[ht]
\centering
\begin{tabular}{rrrrr}
  \hline
 & Estimate & Std. Error & t value & Pr($>$$|$t$|$) \\ 
  \hline
(Intercept) & 0.4775 & 0.2123 & 2.25 & 0.0546 \\ 
  x & 0.0102 & 0.0299 & 0.34 & 0.7417 \\ 
   \hline
\end{tabular}
\caption{Here is the table we made} 
\end{table}\end{Schunk}

\pagebreak 
\section{How to typeset \textsf{R} code}
\pagebreak 
\section{How to typeset \textsf{R} code}


\end{document}

